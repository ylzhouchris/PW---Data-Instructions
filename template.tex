\documentclass{article}


\usepackage{arxiv}

\usepackage[utf8]{inputenc} % allow utf-8 input
\usepackage[T1]{fontenc}    % use 8-bit T1 fonts
\usepackage{hyperref}       % hyperlinks
\usepackage{url}            % simple URL typesetting
\usepackage{booktabs}       % professional-quality tables
\usepackage{amsfonts}       % blackboard math symbols
\usepackage{nicefrac}       % compact symbols for 1/2, etc.
\usepackage{microtype}      % microtypography
\usepackage{lipsum}
\usepackage{todonotes}

\title{Breeze Data Instructions}


\author{
  Y. Zhou\thanks{Use footnote for providing further
    information about author (webpage, alternative
    address)---\emph{not} for acknowledging funding agencies.} \\ Senseable City Laboratory \\
    Massachusetts Institute of Technology\\ Cambridge, MA, 02138 \\ \& \\
  Department of Geography and Resource Management\\
  The Chinese University of Hong Kong\\
  New Territories, Hong Kong, 999077 \\
  \texttt{yulun.zhou@link.cuhk.edu.hk} \\}
  %% examples of more authors
   
  %% \AND
  %% Coauthor \\
  %% Affiliation \\
  %% Address \\
  %% \texttt{email} \\
  %% \And
  %% Coauthor \\
  %% Affiliation \\
  %% Address \\
  %% \texttt{email} \\
  %% \And
  %% Coauthor \\
  %% Affiliation \\
  %% Address \\
  %% \texttt{email} \\


\begin{document}
\maketitle

\begin{abstract}
Documentation of key information to use the organized Breeze dataset. 
\end{abstract}



%% How To search for help: LATEX, bold font 


% keywords can be removed

\tableofcontents

\newpage
\section{Brief Data Information}
\subsection{Data Source}
The Breeze dataset was retrieved by the MIT senseable city laboratory (SCL) from the Run-Keeper company in 2016. The original dataset contains all human trajectories collected from the Breeze app in the city of Boston and San Francisco from Apr.2014 to May.2015.

Please refer to the Supplementary Material of the cognitive load paper for descriptive statistics of the dataset. 

\section{The Organized Dataset - Updated 19/5/17}

\paragraph{Path Types} - The cleaned dataset mostly include 3 types of paths:
\begin{itemize}
    \item Raw GPX Paths ($-rawGPX.pk$) - raw paths collected, a list of coordinates gathered from GPX files. 
    \item Map-matched Paths ($-match.pk$) - map-matched paths using the raw paths and the footpath network. 
    \item Shortest Distance Paths ($-short.pk$) - shortest distance paths calculated using the same OD pairs from the raw GPX paths and the foot path network. 
\end{itemize}

\paragraph{Data Cleaning Objectives} 
\begin{enumerate}
  \item Map-match the raw GPS trajectories to the network. 
  \item Remove paths irrelevant to the proposed analysis. (Running paths/Hang-arounds)
\end{enumerate}

\paragraph{Data Loss} - The cleaned dataset has inevitably experienced significant data loss through the entire cleaning process. Major sources of data loss include map-matching failures, too short paths, and running paths, which are irrelevant to the proposed analysis. Taken the loss, we still get around 40\% of all data, the number of which is still above 10K. 

Main sources of data loss include:
\begin{enumerate}
  \item (Major) Map-Matching Failures (failure and poor matches).
  \item Irrelevant Paths. 
        \begin{itemize}
            \item (Major) Too Short Paths. 
            \item Hang-around Paths
            \item Running Paths
            \item Too Long Paths
        \end{itemize}
\end{enumerate}

More details of the data cleaning procedure will be introduced in the next section.

\paragraph{Representation of Paths} - The original paths came as a list of coordinates. For the convenience of calculation, we proposed other representations as well:
\begin{itemize}
    \item List of Graph Node Indices
        \begin{itemize}
            \item Original Graph
            \item Simplified Graph
        \end{itemize}
    \item NOTICE: The origin and destination of paths could be coordinates since they are not close (defined as direct link distance<30m) to any nodes in the graph
\end{itemize}




\section{Data Cleaning Details - Updated 19/5/17}
\subsection{The Overall Procedure}
\begin{enumerate}
    \item Remove Running Paths

    Understandably, we excluded all running paths using the information provided by the Breeze app. 

    \item Map-Matching.
    A HMM-based map-matching algorithm was applied to raw-GPX trajetories using the GraphHoper API and walk paths provided by Open Street Map.
    
    * Coordinates in the map-matched trajectories were linked to vertices in the walking graph extracted from GraphHoper with an error tolerance of 5m. 
    
    \paragraph{Map-Matching Problems}
    \begin{enumerate}
        \item Map-Matching Failures - About 10k paths failed in the map-matching, possibly due to missing links in the walking network extracted from the Open Street Map. 
            
        \item The Origin-Destination Displacement Problem
            \begin{itemize}
                \item We observed a severe displacement between the origin and destination in the map-matched trajectories as direct outputs from the GraphHoper API. The displacement can be as large as hundreds of meters, which significantly changed the geometry of human paths. More severely, the differences in OD locations lead to very different shortest paths and so make the following comparison between human paths and shortest paths unfair. 
                
                \item Solution: The OD Trim Algorithm.
                
                We proposed a quick fix to the OD Displacement problems. By trimming the extra segments in the map-matched paths around the origin and destination of the raw paths, we intend to fix the majority of the problems. 
                
                \item Data Loss:
                    \begin{itemize}
                        \item Off-track paths: Do>150 (k0) | Dd>150 (k1).
                        \item Less than 2 Nodes after OD trimming (k2).  
                    \end{itemize}
            \end{itemize}    
            
            \item Poor Matches
            \begin{itemize}
                \item Some of the paths were map-matched poorly to the network, either due to high-density and confusing candidate links in the network or due to large GPS jitters in certain areas of the city especially those with high-rise buildings affecting GPS signals.
            
                \item We measured the Hausdorff distance ($HD$) between the raw path and map-matched path as an indicator of map-matching quality. We took the minimum value of Hausdorff distance in both directions. 
            
                \item Data Loss - We exluded all paths having $HD>60m$ as poor matches. The threshold was determined by repeated visualization and comparison. This is the major source of data loss in the entire data cleaning process (about 60k in Boston) mainly due to there being many off-the-network paths in the original dataset, including many indoor movements. These off-the-network paths should be excluded. Also, other poorly-matched paths should be excluded since the proposed analysis is about detailed geometric properties in human paths, such as turns, which is very sensitive to very minor changes in the paths.   
        
            \end{itemize}
        \end{enumerate}
        
    \item Too Short Paths - We excluded all paths having their shortest network distance shorter than 150m.
    
    \item Too Long Paths - We excluded all paths having their shortest network distances longer than 8000m due to small sample size. 

    \item Loops - We excluded all paths containing repeated nodes.
    
    \item Hang-around Paths
    
    \item Other Data Cleaning Steps
        \begin{itemize}
            \item Remove consecutive duplicates
        \end{itemize}
\end{enumerate}


\subsection{The OD Displacement Problem}

The problem was caused by the way the HMM-based map-matching algorithm, in which the closest edge in graph to the O and D was first identified as the start of the searching process for the map-matched path. In doing so, such a problem can be caused. Our practice shows that this problem happens in most of the trajectories and cannot be ignored since it affects the geometry and the related turning properties significantly.
            
+ Distribution of the OD displacements.

+ Examples 

Through a large number of visualizations and comparison, we classified all paths into two sub-categories, totally-off-track paths and slightly-off-track paths. The categorization is based on the direct great circle distance between the O in the raw and map-matched paths (Do) and the corresponding distance between Ds (Dd).

Paths having Do>=t 

\paragraph{Calculate OD Displacement.} First, we calculate the Origin Destination displacement of all paths that are successfully map-matched ($codes/e2-1-calc-od-displacement.ipynb$).  

Algorithm Logic:
\begin{enumerate}
  \item Read all raw paths (R) and map-matched paths (M).
  \item Get O and D of R (Ro, Rd) and M (Mo, Md).
  \item Calculate the great circle distances Do = great-circle($R_o$, $M_o$) and Dd = great-circle(Rd, Md).
\end{enumerate}

\paragraph{Analyze Do and Dd} We first checked the distribution of Do and Dd, both follow heavy-tailed distribution and could be as large as thousands of meters.  

By visualizing a large number of paths having different Do and Dd values, we found that paths having Do and Dd larger than a threshold t were Off-Track Paths, usually indoors or within a large open space. The Off-Track paths are beyond the scope of this study and so should be excluded. 

Paths with Do and Dd smaller than the threshold t were On-Track paths subject to the problem of having extra segments at the start or end of the paths, which could be corrected by applying an OD Trim Algorithm (explained later). 

To identify the appropriate t, we visualized 50-100 paths respectively for t=50, t=100, t=150, and t=200, and selected t=150 as the threshold values. 

\paragraph{Off-Track Paths}
We removed all Off-Track Paths having Do>t or Dd>t. 

\paragraph{On-Track Paths and OD Trim Algorithm}
We identified paths having both Do<=t ($data/*-clean-k0.pk$)  and Dd<=t (*-clean-k1.pk) as On-Track Paths and applied the OD trim algorithm to all On-Track Paths. 

OD Trim Algorithm (codes/s2-2-OD-Trim.ipynb)


A small proportion of On-Track paths shrank to 0 or 1 nodes after applying the OD Trim algorithm. We excluded these paths. All remaining paths are stored in $data/*-clean-k012-*.pk$.


\subsection{Map-Matching Quality Check}
In the previous check, we calculate the hausdorff distance between the raw paths and the map-matched paths. However, the calculated distance were affected by the OD displacement problem. Now as the OD disaplcement problem is fixed. We re-calculate the hausdorff distance between raw GPX and the OD-Trimed paths as a indicator for map-matching quality. 


\subsection{Exclude Paths with too short OD Direct Link}
During the data cleaning process, we observed many paths having loops which are not suitable for our proposed comparison between human paths and shortest paths since the shortest paths will be very short for loop and so making an unfair comparison. Paths having loops need to be excluded. 

Paths having loops are usually identified with a short direct distance between OD. So we first calculate the direct distance between OD ($D_od$) for all paths that are successfully map-matched and then exclude all paths having $D_od$ < 150m. The threshold of 150m was determined by repeated experiments and visualizations.   


\begin{figure}
  \centering
  \includegraphics[width=.6\textwidth]{figures/ShortODDirectLink.PNG}
  \caption{Paths with a very short OD Direct Link. The red line indicates the raw GPS path while the blue line indicates the map-matched path after OD trim.}
  \label{fig:fig1}
\end{figure}






\subsection{Other approaches that can be applied}

\paragraph{Time Smoothing to the raw trajectory}
Eliminate Jumps, Improve Map-Matching Quality

\paragraph{Time Sparsing to the raw trajectory}
According to the map-matching paper. 

\paragraph{Jumps/Over-Speed Filtering in raw trajectories}
Average Speed and Temporal Speed 




\section{Calculation of the shortest path for ODs not matched to graph nodes}
The origin and destination of map-matched paths were biased due to the OD displacement problem caused by the map-matching algorithm, so we need to re-calculate the shortest paths using corrected O and D. During the calculation, we also need a special mechanism to deal with the OD on link problem, that is, OD can not be aggregated to intersections. 

\paragraph{Input: Network}
We have two options, one is to use the network we extracted from Graphhoper map-matching algorithm, and the other is to use the simplified one from Networkx. Since the map-matched algorithm was conducted based on the previous one, we also use the previous one ($graph-original['sub']$) for calculating the shortest path.

* Whether to simply it using NetworkX

\paragraph{Input: OD}
We only process the paths having either O or D apart from intersections. By apart, we mean more than 30 meters away. We get XX and XX such paths in BOS and SF. 

\paragraph{Algorithm for OD apart from intersections}
(If O of path P is not near intersection)
\begin{enumerate}
    \item Get the neighbor p to O on P
    \item Get all links L connected to p in the graph G
    \item Find l in L that is closest to p (perpendicular distance)
    \item Get the two nodes of l, namely $O_0$ and $O1$
    (Check if OD are closest to the same link, if so, put on side.)
    \item (If D is not near intersection) Repeat the process and get $D_0$ and $D_1$
    \item Calculate all shortest paths between all combinations of Os and Ds
    \item Calculate the distances of all calculated shortest distance paths and add the distance between real O and D and O's 
    \item Get the path between OD having the shortest distance as the shortest path.  
\end{enumerate}








\end{document}
